\documentclass[a4paper]{report}
\usepackage[utf8]{inputenc}
\usepackage[numbers]{natbib}
\usepackage{hyperref}
\usepackage{graphicx}
\usepackage{datetime}
\usepackage{color}
\usepackage{microtype}
\usepackage{draftwatermark}

\SetWatermarkScale{5}

%% Nicely format and linebreak URLs in the bibliography (and elsewhere).
\usepackage{url}
%% Define a new 'leo' style for the package that will use a smaller font.
\makeatletter
\def\url@leostyle{%
  \@ifundefined{selectfont}{\def\UrlFont{\sf}}{\def\UrlFont{\small\ttfamily}}}
\makeatother
%% Now actually use the newly defined style.
\urlstyle{leo}

%% Nicer formatting of figure captions.
\usepackage[font=small,format=plain,labelfont=bf,up,textfont=it,up]{caption}

%\newcommand{\hi}[1]{{\color{red}\tiny \em #1\/}\\}
\newcommand{\todo}[1]{\footnote{{\color{red} {\bf TODO:} #1}}}

% ------------------------------------------------------------------------------
% Metadata
% ------------------------------------------------------------------------------
\title{Ontology Learning from Swedish Text}

\author{Jan Daniel Bothma}

\begin{document}

\maketitle

\abstract{}

\tableofcontents

\chapter{Introduction}

In this chapter we will introduce the concept of Ontology Learning (OL), briefly identify some recent and significant work in OL and summarize open research questions identified in our literature review.
We will then propose the objective of this thesis along with research questions, how we intend to answer those questions, and delimitations to clarify the scope of this thesis. Finally, we will describe the layout of the remainder of this report.

\section{Background}

During the past decade, the Semantic Web has emerged as a promising evolution of the human readable web. By adding formal semantics in the form of ontologies to data on the web, it can be consumed and utilized by software agents and services in an automatic fashion.
\todo{what are ontologies used for more?}

\todo{knowledge acquisition bottleneck}.

The field of Ontology Learning has evolved to try and tackle this problem.
Ontology Learning commonly involves some combination of preprocesing, term extraction, concept formation, concept hierarchy induction, non-taxonomic relation learning, axiom extraction and construction of the ontology from these extracted elements in a form suitable for its application.
\todo{Eva's explanation connecting with evaluation and OE is intuitively more realistic than Ciamiano's disconnected explanation}
\todo{explain more + graphic + cite}

\todo{give some examples of well-evaluated and newer techniques and their evaluation}

\section{Problem}

While many techniques for extracting terms and concepts have matured and have been evaluated against various competing techniques and in various settings, many areas of research are still somewhat unexplored.
Additionally, new technologies and online communities are providing new opportunities for supporting ontology learning.

Many existing methods rely on domain-specific models which need to be trained.
This training often involves manually annotated corpora which are time-consuming to produce, require domain expertise and aren't error-free.
There has been relatively little research in combining evidence from various languages.

\section{Objective}

The objective of this thesis is to support ongoing research in Ontology Learning by researching the high level requirements for ongoing ontology learning research and application, and by applying a limited number existing methods to learning domain ontologies from Swedish text.
\cite{expand and clarify}

\section{Research Questions}

\begin{itemize}
  \item{What are the requirements on a general Ontology Learning research and application}
\end{itemize}

\section{Approach}

\section{Delimitations}

\section{Report Structure}


\chapter{Current Techniques}


\chapter{Evaluation Method}


\chapter{Evaluation Results}


\chapter{Conclusion}


\chapter{Further Work}

\clearpage

\addcontentsline{toc}{chapter}{References}

\renewcommand\bibname{References}
\bibliographystyle{unsrtnat}
\bibliography{report}

\end{document}
